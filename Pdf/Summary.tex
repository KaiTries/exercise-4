\documentclass[11pt]{article}

\usepackage{geometry}
\geometry{a4paper, margin=1in}

\usepackage{times} % Times New Roman font
\usepackage{hyperref} % For hyperlinks
\usepackage{graphicx} % For including images
\usepackage{listings}
\usepackage{textcomp}

\usepackage{xcolor} % Required for setting custom colors

\lstset{
  backgroundcolor=\color{lightgray}, % Set the background color
  frame=single, % Adds a frame around the code
  numbers=left, % Line numbers on the left
  numberstyle=\tiny, % Line numbers styling
  basicstyle=\ttfamily\small, % Set the font style
  keywordstyle=\color{blue}, % Set the color for keywords
  commentstyle=\color{green}, % Set the color for comments
  stringstyle=\color{red}, % Set the color for strings
  breaklines=true, % Wrap long lines
  showstringspaces=false, % Don't show spaces in strings as special character
}


\title{Exercise 5}
\author{Kai Schultz}
\date{March 18 2024} % Or specify a date

\begin{document}

\maketitle

\section*{Task 1: Take part to a decentralised social network}
\subsection*{Task 1.1: Update your FOAF Profile}
The Url to my Card: \href{https://wiser-solid-xi.interactions.ics.unisg.ch/kai/profile/card#me}{/kai/card\#me} \\
I added my Name and my mail address and you and Jonas as people that I know.
\begin{lstlisting}
@prefix foaf: <http://xmlns.com/foaf/0.1/>.
@prefix solid: <http://www.w3.org/ns/solid/terms#>.

<>
    a foaf:PersonalProfileDocument;
    foaf:maker <https://wiser-solid-xi.interactions.ics.unisg.ch/kai/profile/card#me>;
    foaf:primaryTopic <https://wiser-solid-xi.interactions.ics.unisg.ch/kai/profile/card#me>.

<https://wiser-solid-xi.interactions.ics.unisg.ch/kai/profile/card#me>
    
    solid:oidcIssuer <https://wiser-solid-xi.interactions.ics.unisg.ch/>;
    a foaf:Person;
    foaf:name "Kai Schultz";
    foaf:mbox "kai.schultz@student.unisg.ch" ;
    foaf:knows <https://wiser-solid-xi.interactions.ics.unisg.ch/danaivach/profile/card#me> ;
    foaf:knows <https://wiser-solid-xi.interactions.ics.unisg.ch/Jonas/profile/card#me> .
    
\end{lstlisting}

\subsection*{Task 1.2: Query the distributed social graph}
Both queries can also be found as sparql files on Github: \href{https://github.com/KaiTries/exercise-4/blob/main/sparqlNoTraversal.sparql}{nonTraversal} | \href{https://github.com/KaiTries/exercise-4/blob/main/sparqlLinkTraversal.sparql}{Traversal}.\\Use the cli commands comunica-sparql or comunica-sparql-link-traversal
\subsubsection*{1. Query the people you know based on your FOAF profile}
\begin{lstlisting}
"PREFIX foaf: <http://xmlns.com/foaf/0.1/>
SELECT DISTINCT ?person
WHERE {
 <https://wiser-solid-xi.interactions.ics.unisg.ch/kai/profile/card#me> foaf:knows ?person.
}
\end{lstlisting}
The above sparql query searches through all triples on the specified Iri (my profile card). The specified triple that were looking for is: \textless MyWebID\textgreater \textless foaf:knows\textgreater \textless Any Person\textgreater. That is what we did in the above query. From this query we select all distinct people we get.
\subsubsection*{1. Query the names of all people interconnected in the distributed social graph}
\begin{lstlisting}
PREFIX foaf: <http://xmlns.com/foaf/0.1/>
SELECT DISTINCT ?name
WHERE {
  <https://wiser-solid-xi.interactions.ics.unisg.ch/kai/profile/card#me> (foaf:knows|^foaf:knows)+ ?person.
  ?person (foaf:name|foaf:firstName|foaf:givenName|foaf:nickName)+ ?name.
}
\end{lstlisting}
In the above query we use the \href{https://www.w3.org/TR/sparql11-query/\#propertypaths}{property paths} to query through the graph. The Comunica Link Traversal engine enables us to go through the external nodes and that is how we can crawl  through the entire social graph.\\
While the same syntax is still possible without the link traversal we will in our case not be able to profit from it, since we do not execute any http requests to the found persons. If there were transitive relationships on on graph, so e.g.:\\
a owl:sameas :b.\\
:b owl:sameas :c.\\
We could do:
\begin{lstlisting}
PREFIX owl: <http://www.w3.org/2002/07/owl#>
SELECT *
WHERE
{
  ?x owl:sameAs+ ?y
}
\end{lstlisting}
This is an exampele of how a navigational query could be used on a single graph. So all in all there is no difference in syntax between the two queries since they are both just sparql. The executed HTTP requests differ, since one cli command can traverse link and the other cant. So executing it with comunica-sparql will result in 1 HTTP request and executing it with comunica-sparql-link-traversal will result in querying the entire reachable graph.

\section*{Task 2: Build an application for your Solid pod}
All the code for the task can be found on Github: \href{https://github.com/KaiTries/exercise-4/blob/main/src/env/solid/Pod.java}{/src/env/solid/Pod.java}\\
Code should be executed as described in the Exercise  description.
\end{document}
